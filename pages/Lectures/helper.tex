\documentclass[]{article}
\usepackage{lmodern}
\usepackage{amssymb,amsmath}
\usepackage{ifxetex,ifluatex}
\usepackage{fixltx2e} % provides \textsubscript
\ifnum 0\ifxetex 1\fi\ifluatex 1\fi=0 % if pdftex
  \usepackage[T1]{fontenc}
  \usepackage[utf8]{inputenc}
\else % if luatex or xelatex
  \ifxetex
    \usepackage{mathspec}
  \else
    \usepackage{fontspec}
  \fi
  \defaultfontfeatures{Ligatures=TeX,Scale=MatchLowercase}
\fi
% use upquote if available, for straight quotes in verbatim environments
\IfFileExists{upquote.sty}{\usepackage{upquote}}{}
% use microtype if available
\IfFileExists{microtype.sty}{%
\usepackage{microtype}
\UseMicrotypeSet[protrusion]{basicmath} % disable protrusion for tt fonts
}{}
\usepackage[margin=1in]{geometry}
\usepackage{hyperref}
\hypersetup{unicode=true,
            pdfborder={0 0 0},
            breaklinks=true}
\urlstyle{same}  % don't use monospace font for urls
\usepackage{graphicx,grffile}
\makeatletter
\def\maxwidth{\ifdim\Gin@nat@width>\linewidth\linewidth\else\Gin@nat@width\fi}
\def\maxheight{\ifdim\Gin@nat@height>\textheight\textheight\else\Gin@nat@height\fi}
\makeatother
% Scale images if necessary, so that they will not overflow the page
% margins by default, and it is still possible to overwrite the defaults
% using explicit options in \includegraphics[width, height, ...]{}
\setkeys{Gin}{width=\maxwidth,height=\maxheight,keepaspectratio}
\IfFileExists{parskip.sty}{%
\usepackage{parskip}
}{% else
\setlength{\parindent}{0pt}
\setlength{\parskip}{6pt plus 2pt minus 1pt}
}
\setlength{\emergencystretch}{3em}  % prevent overfull lines
\providecommand{\tightlist}{%
  \setlength{\itemsep}{0pt}\setlength{\parskip}{0pt}}
\setcounter{secnumdepth}{0}
% Redefines (sub)paragraphs to behave more like sections
\ifx\paragraph\undefined\else
\let\oldparagraph\paragraph
\renewcommand{\paragraph}[1]{\oldparagraph{#1}\mbox{}}
\fi
\ifx\subparagraph\undefined\else
\let\oldsubparagraph\subparagraph
\renewcommand{\subparagraph}[1]{\oldsubparagraph{#1}\mbox{}}
\fi

%%% Use protect on footnotes to avoid problems with footnotes in titles
\let\rmarkdownfootnote\footnote%
\def\footnote{\protect\rmarkdownfootnote}

%%% Change title format to be more compact
\usepackage{titling}

% Create subtitle command for use in maketitle
\providecommand{\subtitle}[1]{
  \posttitle{
    \begin{center}\large#1\end{center}
    }
}

\setlength{\droptitle}{-2em}

  \title{}
    \pretitle{\vspace{\droptitle}}
  \posttitle{}
    \author{}
    \preauthor{}\postauthor{}
    \date{}
    \predate{}\postdate{}
  

\begin{document}

output: pdf\_document: default html\_document: default

\hypertarget{ma1}{%
\section{MA(1)}\label{ma1}}

time - \(T\) information set by \(\Omega_T\)

\begin{itemize}
\tightlist
\item
  \(\Omega_T = \{y_T , y_{T −1} , y_{T −2} , ..\},\)
\item
  \(\Omega _T = \{\varepsilon_T , \varepsilon_{T −1} , \varepsilon_{T −2} , \cdots\}.\)
\end{itemize}

Assembling the discussion thus far, we can view the time-T information
set as containing the current and past values of either (or both) \(y\)
and \(\varepsilon,\)
\[\Omega_T =\{ y_{T} , y_{T −1} , y_{T −2}, \cdots, \varepsilon _{T } , \varepsilon_{T −1} , \varepsilon _{T −2} , \cdots.\}\]

\hypertarget{optimal}{%
\section{Optimal}\label{optimal}}

optimal forecast of \(y\) at some future time \(T + h\).

The optimal forecast is the one with the smallest loss on average, that
is, the forecast that minimizes expected loss.

optimal forecast is the conditional mean,
\[\mathrm{E}(y_{T +h} |\Omega_T ),\] In general, the conditional mean
need not be a linear function of the elements of the information set.
Because linear functions are particularly tractable, we prefer to work
with linear forecasts -- forecasts that are linear in the elements of
the information set -- by finding the best linear approximation to the
conditional mean, called the linear projection, denoted
\[P(y_{T +h} |\Omega_T ).\] This explains the common term ``linear least
squares forecast.'' The linear projection is often very useful and
accurate, because the conditional mean is often close to linear. In
fact, in the Gaussian case the conditional expectation is exactly
linear, so that
\[\mathrm{E}(y_{T +h} |\Omega_T ) = P (y_{T +h} |\Omega_T ).\]

Our forecasting method is always the same: we write out the process for
the future time period of interest, \(T + h\), and project it on what's
known at time \(T\) , when the forecast is made. This process is best
learned by example.

\hypertarget{d}{%
\section{d}\label{d}}

Consider an MA(2) process,
\[y_t = \varepsilon_t + \theta_1 \varepsilon_{t−1} + \theta_2 \varepsilon_{t−2},\quad \varepsilon_t ∼ W N (0, \sigma ^2 ).\]
Suppose we're standing at time \(T\) and we want to forecast
\(y_{T +1}\) . First we write out the process for \(T + 1\),
\[y_{T +1} = \varepsilon_{T +1} + \theta_1 \varepsilon _T + \theta_2 \varepsilon_{T −1} .\]

Then we project on the time-T information set, which simply means that
all future innovations are replaced by zeros. Thus
\[y_{ T +1,T} = P (y_{T +1} |\Omega_T ) = \theta_1 \varepsilon_T + \theta_2 \varepsilon_{T −1} .\]

To forecast 2 steps ahead, we note that
\[y_{T +2} = \varepsilon_{T +2} + \theta_1 \varepsilon_{T +1} + \theta _2 \varepsilon_T ,\]
and we project on the time-T information set to get
\[y_{T +2,T} = \theta_2 \varepsilon_T .\] Continuing in this fashion, we
see that \[y_{ T +h,T} = 0,\] for all \(h>2\).

Now let's compute the corresponding forecast errors. We have:
\[e_{T +1,T} = \varepsilon_{T +1} W N\]
\[e_{T +2,T} = \varepsilon _{T +2} + \theta_1 \varepsilon _{T +1}  (M A(1))\]
\[e_{ T +h,T} = \varepsilon_{T +h}  + \theta _1 \varepsilon_{T +h−1} + \theta_2 \varepsilon
_{T +h−2} (M A(2)),\] for all \(h>2\).

Finally, the forecast error variances are: \[\sigma_1^2 = \sigma^2\]
\[\sigma_2^2 = \sigma_2 (1 + \theta_1^2 )\]
\[\sigma_h^2  = \sigma_2 (1 + \theta_1^2 + \theta_2^2 ),\] or all h
\textgreater{} 2. Moreover, the forecast error variance for \(h>2\) is
just the unconditional variance of \(y_t\).

Now consider the general M A(q) case. The model is
\[y_t = \varepsilon_ t + \theta_ 1 \varepsilon_{t-1} + \cdots +\theta_ q \varepsilon_{t−q}.\]
First, consider the forecasts. If \(h\leq\), the forecast has the form
\[y_{ T +h,T} = 0 + \text{"adjustment"}\] whereas if \(h>q\) the
forecast is \[y_{ T +h,T} = 0.\]

Thus, an M A(q) process is not forecastable (apart from the
unconditional mean) more than \(q\) steps ahead. All the dynamics in the
M A(q) process, which we exploit for forecasting, ``wash out'' by the
time we get to horizon q, which reflects the autocorrelation structure
of the M A(q) process. (Recallthat, as we showed earlier, it cuts off at
displacement q.) Second, consider the corresponding forecast errors.
They are \[e_{ T +h,T} = M A(h − 1)\] for \(h \leq q\) and
\[e_{ T +h,T} = M A(q)\] for h \textgreater{} q. The h-step-ahead
forecast error for \(h > q\) is just the process itself, minus its mean.

Finally, consider the forecast error variances. For h \leq q,
\[\sigma_h^2  \leq var(y t ),\] whereas for \(h > q\),
\[\sigma_h^2  = var(y t ).\] In summary, we've thus far studied the M
A(2), and then the general M A(q), process, computing the optimal
h-step-ahead forecast, the corresponding forecast error, and the
forecast error variance. As we'll now see, the emerging patterns that we
cataloged turn out to be quite general.

\hypertarget{optimal-point-forecasts-for-infinite-order-moving-averages}{%
\section{Optimal Point Forecasts for Infinite-Order Moving
Averages}\label{optimal-point-forecasts-for-infinite-order-moving-averages}}

By now you're getting the hang of it, so let's consider the general case
of an infinite-order M A process. The infinite-order moving average
process may seem like a theoretical curiosity, but precisely the
opposite is true. Any covariance stationary process can be written as a
(potentially infinite-order) moving average process, and moving average
processes are easy to understand and manipulate, because they are
written in terms of white noise shocks, which have very simple
statistical properties. Thus, if you take the time to understand the
mechanics of constructing optimal forecasts for infinite moving-average
processes, you'll understand everything, and you'll have some powerful
technical tools and intuition at your command. Recall that the general
linear process is Recall that the general linear process is
\[y_t=\sum_{i=0}^\infty b_i\varepsilon_{t-i}, \quad \varepsilon_t\sim WN(0,\sigma^2)\]
\[y_{T +h} = \varepsilon_{T +h} + b_1 \varepsilon_{T +h−1} + \cdots + b_h \varepsilon_T + b_{h+1} \varepsilon_{T −1} + \cdots\]

Then we project \(y_{T +h}\) on the time-\(T\) information set. The
projection yields zeros for all of the future \(\varepsilon\)'s (because
they are white noise and hence unforecastable), leaving
\[y_{ T +h,T} = b_h \varepsilon_T + b_{h+1} \varepsilon _{T −1} + \cdots\]
It follows that the \(h\)-step ahead forecast error is serially
correlated; it follows an MA(h − 1) process,
\[e_{ T +h,T} = (y_{T +h} − y_{ T +h,T} ) =\sum_{i=0}^\infty b_i\varepsilon_{t-i}, \quad \varepsilon_t\]

with mean 0 and variance
\[\sigma _h^2=\sigma ^ 2\sum_{i=0}^{h-1} b_i^2.\]

A number of remarks are in order concerning the optimal forecasts of the
general linear process, and the corresponding forecast errors and
forecast error variances. First, the 1-step-ahead forecast error is
simply \(\varepsilon T +1\) . \(\varepsilon T +1\) is that part of
\(y T +1\) that can't be linearly forecast on the basis of \(\Omega_t\)
(which, again, is why it is called the innovation). Second, although it
might at first seem strange that an optimal forecast error would be
serially correlated, as is the case when h \textgreater{} 1, nothing is
awry. The serial correlation can't be used to improve forecasting
performance, because the autocorrelations of the M A(h−1) process cut
off just before the beginning of the time-T information set \$
\varepsilon T , \varepsilon T −1 , \cdots. \$This is a general and
tremendously important property of the errors associated with optimal
forecasts: errors from optimal forecasts can't be forecast using
information available when the forecast was made.

If you can forecast the forecast error, then you can improve the
forecast, which means that it couldn't have been optimal. Finally, note
that as h approaches infinity \(y_{ T +h,T}\) approaches zero, the
unconditional mean of the rocess, and \(a\), the unconditional variance
of the process, which reflects the fact that as h approaches infinity
the conditioning information on which the forecast is based becomes
progressively less useful. In other words, the distant future is harder
to forecast than the near future!

Now we construct interval and density forecasts. Regardless of whether
the moving average is finite or infinite, we proceed in the same way, as
follows. The definition of the h-step-ahead forecast error is
\[e_{ T +h,T} = y_{T +h} − y_{ T +h,T} .\] Equivalently, the
h-step-ahead realized value, \(y_{T +h} ,\) equals the forecast plus the
error, \[y_{T +h} = y_{ T +h,T} + e_{ T +h,T} .\] If the innovations are
normally distributed, then the future value of the series of interest is
also normally distributed, conditional upon the information set
available at the time the forecast was made, and so we have the 95\%
h-step- ahead interval forecast \(y_{ T +h,T} \pm 1.96\sigma_h\) . In
similar fashion, we construct the h-step-ahead density forecast as
\[N (y_{ T +h,T} , \sigma_h^2  )\]. The mean of the conditional
distribution of \(y_{T +h}\) is \(y_{ T +h,T}\) , which of course must
be the case because we constructed the point forecast as the conditional
mean, and the variance of the conditional distribution is \(\sigma_h^2\)
, the variance of

As an example of interval and density forecasting, consider again the M
A(2) process,
\[y_t = \varepsilon_t + \theta_1 \varepsilon_{t-1} + \theta_2 \varepsilon_{t−2}
\varepsilon_t ∼ W N (0, \sigma_2 ).\] Assuming normality, the
1-step-ahead 95\% interval forecast is
\[y_{T +1,T} = (\theta_1 \varepsilon_T + \theta_2 \varepsilon_{T −1} ) \pm 1.96\sigma ,\]
and the 1-step-ahead density forecast is
\[N (\theta_1 \varepsilon_T + \theta_2 \varepsilon_{T −1} , \sigma_2 ).\]

\hypertarget{making-the-forecasts-operational}{%
\section{Making the Forecasts
Operational}\label{making-the-forecasts-operational}}

So far we've assumed that the parameters of the process being forecast
are known. In practice, of course, they must be estimated. To make our
forecasting procedures operational, we simply replace the unknown
parameters in our formulas with estimates, and the unobservable
innovations with residuals. Consider, for example, the M A(2) process,
\[y_t = \varepsilon_t + \theta_1 \varepsilon_{t-1} + \theta_2 \varepsilon_{t−2} .\]

As you can readily verify using the methods we've introduced, the 2-step
ahead optimal forecast, assuming known parameters, is
\[y_{T +2,T} = \theta_2 \varepsilon _T ,\] with corresponding forecast
error
\[e_{T +2,T} = \varepsilon_{T +2} + \theta_ 1 \varepsilon _{T +1} ,\]
and forecast-error variance
\[\sigma _2^2 = \sigma_2 (1 + \theta_1^2 ).\]

To make the forecast operational, we replace unknown parameters with
estimates and the time-T innovation with the time-T residual, yielding
\[\hat y_{T +2,T} = \hat\theta_2 \hat\varepsilon_T\] and forecast error
variance \[\hat\sigma_2^2 = \hat\sigma^2 (1 + \hat\theta_1^2 ).\] Then,
if desired, we can construct operational 2-step-ahead interval and den-
sity forecasts, as
\[\hat y_{T +2,T} \pm \frac{z_\alpha}{2 \hat\sigma^2}\] and
\[N (\hat y_{T +2,T} ,\hat\sigma^2 ).\]

The strategy of taking a forecast formula derived under the assumption
of known parameters, and replacing unknown parameters with estimates, is
a natural way to operationalize the construction of point forecasts.
However, using the same strategy to produce operational interval or
density forecasts involves a subtlety that merits additional discussion.
The forecast error variance estimate so obtained can be interpreted as
one that ignores parameter estimation uncertainty, as follows.

Recall once again that the actual future value of the series is
\[y_{T+2} = \varepsilon_{T+2} + \theta_1 \varepsilon_{ T +1} + \theta_2 \varepsilon_ T ,\]
and that the operational forecast is
\[\hat y_{T +2,T} = \hat \theta_2 \varepsilon_T .\] Thus the exact
forecast error is
\[\hat e_{T +2,T} = y_{T+2} − \hat y_{T +2,T} = \varepsilon_{T+2} + \theta_1 \varepsilon_{T +1} + (\theta_2 -\hat \theta_2 )\varepsilon_ T ,\]
the variance of which is very difficult to evaluate.

So we make a convenient approximation: we ignore parameter estimation
uncertainty by assuming that estimated parameters equal true parameters.
We therefore set \((\theta_2 -\hat \theta_2 ))\) to zero, which yields
\[\hat e_{T +2,T}= \varepsilon_{T+2} + \theta_ 1 \varepsilon_{T +1} ,\]
with variance \[\sigma _2^2 = \sigma_2 (1 + \theta_1^2 ),\] which we
make operational as
\(\hat\sigma _2^2 = \hat\sigma ^2 (1 + \hat \theta^2).\)

\hypertarget{forecasting-cycles-from-an-ar-wolds-chain-rule}{%
\section{Forecasting Cycles From an AR: Wold's Chain
Rule}\label{forecasting-cycles-from-an-ar-wolds-chain-rule}}

\#\#Point Forecasts of Autoregressive Processes

Because any covariance stationary AR(p) process can be written as an
infinite moving average, there's no need for specialized forecasting
techniques for autoregressions. Instead, we can simply transform the
autoregression into a moving average, and then use the techniques we
developed for forecasting moving averages. It turns out, however, that a
very simple recursive method for computing the optimal forecast is
available in the autoregressive case. The recursive method, called the
chain rule of forecasting, is best learned by example.

Consider the AR(1) process,
\[y_ t = \phi y_{t−1} + \varepsilon_ t,\quad \varepsilon_ t ∼ W N (0, \sigma_2 ).\]First
we construct the optimal 1-step-ahead forecast, and then we construct
the optimal 2-step-ahead forecast, which depends on the optimal
1-step-ahead forecast, which we've already constructed. Then we
construct the optimal 3-step-ahead forecast, which depends on the
already-computed 2-step-ahead forecast, which we've already constructed,
and so on.

To construct the 1-step-ahead forecast, we write out the process for
time \(T + 1\), \[y T +1 = \phi y T + \varepsilon T +1 .\] Then,
projecting the right-hand side on the time-T information set, we obtain
\[y_{T +1,T} = \phi y T .\] Now let's construct the 2-step-ahead
forecast. Write out the process for time T + 2,
\[y_{T+2} = \phi y T +1 + \varepsilon_{T+2} .\] Then project directly on
the time-T information set to get \[y{T +2,T} = \phi y_{T +1,T} .\]

Note that the future innovation is replaced by 0, as always, and that we
have directly replaced the time T +1 value of y with its
earlier-constructed optimal forecast. Now let's construct the
3-step-ahead forecast. Write out the process for time T + 3,
\[y T +3 = \phi y_{T+2} + \varepsilon T +3 .\] Then project directly on
the time-T information set, \[y T +3,T = \phi y{T +2,T} .\]

The required 2-step-ahead forecast was already constructed. Continuing
in this way, we can recursively build up forecasts for any and all
future periods. Hence the name ``chain rule of forecasting.'' Note that,
for the AR(1) process, only the most recent value of y is needed to
construct optimal forecasts, for any horizon, and for the general AR(p)
process only the p most recent values of y are needed.

\hypertarget{point-forecasts-of-arma-processes}{%
\subsection{Point Forecasts of ARMA
processes}\label{point-forecasts-of-arma-processes}}

Now we consider forecasting covariance stationary ARMA processes. Just
as with autoregressive processes, we could always convert an ARMA
process to an infinite moving average, and then use our
earlier-developed methods for forecasting moving averages. But also as
with autoregressive processes, a simpler method is available for
forecasting ARMA processes directly, by combining our earlier approaches
to moving average and autoregressive fore- casting.

As always, we write out the ARM A(p, q) process for the future period of
interest,
\[y_{T +h} = \phi_ 1 y_{T +h−1} + \cdots + \phi_ p y_{T +h−p} + \varepsilon_{T +h} + \theta_ 1 \varepsilon_{T +h−1} + \cdots + \theta_ q \varepsilon_ {T +h−q} .\]
On the right side we have various future values of y and
\(\varepsilon\), and perhaps also past values, depending on the forecast
horizon. We replace everything on the right-hand side with its
projection on the time-T information set.

That is, we replace all future values of y with optimal forecasts (built
up recursively using the chain rule) and all future values of
\(\varepsilon\) with optimal forecasts (0), yielding
\[y_{ T +h,T} = \phi _1 y_{T +h−1,T} + \cdots + \phi_p y_{T+h−p,T} + \varepsilon_{ T +h,T} + \theta_1 \varepsilon_{T +h−1,T }+ \cdots + \theta_q \varepsilon_{T +h−q,T} .\]
When evaluating this formula, note that the optimal time-T ``forecast''
of any value of y or \(\varepsilon\) dated time T or earlier is just y
or \(\varepsilon\) itself. As an example, consider forecasting the ARM
A(1, 1) process,
\[y t = \phi y t−1 + \varepsilon t + \theta\varepsilon_{t-1}
\varepsilon t ∼ W N (0, \sigma_2 ).\]

Let's find \(y_{T +1,T}\) . The process at time T + 1 is
\[y T +1 = \phi y T + \varepsilon T +1 + \theta\varepsilon T .\]
Projecting the right-hand side on \(\Omega_T\) yields
\[y_{T +1,T} = \phi y _T + \theta\varepsilon _T .\] Now let's find
\(y{T +2,T}\) . The process at time \(T + 2\) is
\[y_{T+2} = \phi y _{T +1} + \varepsilon_{T+2} + \theta\varepsilon _{T +1 }.\]
Projecting the right-hand side on \(\Omega_T\) yields
\[y{T +2,T} = \phi y_{T +1,T} .\] Substituting our earlier-computed
1-step-ahead forecast yields
\[y_{T +2,T} = \phi (\phi y_ T + \theta\varepsilon_ T ) 
= \phi^2 y _T + \phi\theta\varepsilon _T .\] Continuing, it is clear
that \[y_{T +h,T} = \phi y_{T +h−1,T },\] for all \(h > 1\).

\hypertarget{interval-and-density-forecasts}{%
\section{Interval and Density
Forecasts}\label{interval-and-density-forecasts}}

The chain rule, whether applied to pure autoregressive models or to ARMA
models, is a device for simplifying the computation of point forecasts.
Interval and density forecasts require the h-step-ahead forecast error
variance, which we get from the moving average representation, as
discussed earlier. It is \[\sigma_h^2 
= \sigma_2\sum_{i=0}^\infty b_i^2\] which we operationalize as
\[\hat\sigma_h^2
= \hat\sigma^2
\sum_{i=0}^\infty \hat b_i^2\]

Note that we don't actually estimate the moving average representation;
rather, we solve backward for as many b's as we need, in terms of the
original model parameters, which we then replace with estimates.

Let's illustrate by constructing a 2-step-ahead 95\% interval forecast
for the ARM A(1, 1) process. We already constructed the 2-step-ahead
point forecast, \(y_{T +2,T}\); we need only compute the 2-step-ahead
forecast error variance. The process is
\[y_t = \phi y_{t−1} + \varepsilon_t + \theta\varepsilon_{t-1}\]

Substitute backward for y t−1 to get
\[y_t = \phi(\phi y_{t−2} + \varepsilon_{t-1} + \theta\varepsilon_{t−2 }) + \varepsilon_t + \theta\varepsilon_{t-1}= \varepsilon_t + (\phi + \theta)\varepsilon_{t-1} + \cdots\]
We need not substitute back any farther, because the 2-step-ahead
forecast error variance is \[\sigma _2^2 = \sigma^2 (1+b^2_1 ),\] where
\(b_1\) is the coefficient on \(\varepsilon_{t-1}\) in the moving
average representation of the ARMA(1,1) process, which we just
calculated to be \(( \phi + \theta )\).

Thus the 2-step-pahead interval forecast is
\(y_{T +2,T} \pm 1.96\sigma_2\) , or
\((\phi^2 y_T + \phi\theta\varepsilon_T ) \pm 1.96\sigma\sqrt{ 1 + (\phi + \theta)^2}\)
. We make this operational as


\end{document}
